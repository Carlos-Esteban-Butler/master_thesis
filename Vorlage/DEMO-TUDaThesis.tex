%\documentclass[
%english,
%ruledheaders=section,
%twoside,
%book,
%class=report,
%thesis={type=pp},
%accentcolor=2c,% Auswahl der Akzentfarbe
%custommargins=false,
%marginpar=false,
%BCOR=5mm,%Bindekorrektur, falls notwendig
%parskip=half-,%Absatzkennzeichnung durch Abstand vgl. KOMA-Sript
%fontsize=11pt,
%%	logofile=example-image, %Falls die Logo Dateien nicht vorliegen
%]{tudapub}
\documentclass[
	ngerman,
	ruledheaders=section,%Ebene bis zu der die Überschriften mit Linien abgetrennt werden, vgl. DEMO-TUDaPub
	twoside,
	book,
	class=report,% Basisdokumentenklasse. Wählt die Korrespondierende KOMA-Script Klasse
	thesis={type=Master thesis},% Dokumententyp Thesis, für Dissertationen siehe die Demo-Datei DEMO-TUDaPhd
	accentcolor=9c,% Auswahl der Akzentfarbe
	custommargins=false,% Ränder werden mithilfe von typearea automatisch berechnet
	marginpar=false,% Kopfzeile und Fußzeile erstrecken sich nicht über die Randnotizspalte
	BCOR=5mm,%Bindekorrektur, falls notwendig
	parskip=half-,%Absatzkennzeichnung durch Abstand vgl. KOMA-Sript
	fontsize=11pt,%Basisschriftgröße laut Corporate Design ist mit 9pt häufig zu klein
%	logofile=example-image, %Falls die Logo Dateien nicht vorliegen
]{tudapub}


% Der folgende Block ist nur bei pdfTeX auf Versionen vor April 2018 notwendig
\usepackage{iftex}
\ifPDFTeX
	\usepackage[utf8]{inputenc}%kompatibilität mit TeX Versionen vor April 2018
\fi

%%%%%%%%%%%%%%%%%%%
%Sprachanpassung & Verbesserte Trennregeln
%%%%%%%%%%%%%%%%%%%
\usepackage[ngerman, main=english]{babel}
\usepackage[autostyle]{csquotes}% Anführungszeichen vereinfacht

% Falls mit pdflatex kompiliert wird, wird microtype automatisch geladen, in diesem Fall muss diese Zeile entfernt werden, und falls weiter Optionen hinzugefügt werden sollen, muss dies über
% \PassOptionsToPackage{Optionen}{microtype}
% vor \documentclass hinzugefügt werden.
\usepackage{microtype}
\usepackage{siunitx}
\sisetup{
	output-product=\ensuremath{{\cdot}},
	exponent-product=\ensuremath{\cdot},
	separate-uncertainty = true,
}
%\sisetup{math-micro=\text{µ},text-micro=µ}
\DeclareSIUnit[]\PSL{PSL}
\DeclareSIUnit[]\px{px}
\usepackage{arydshln}
\usepackage{subcaption}
\usepackage{graphicx}
%%%%%%%%%%%%%%%%%%%
%Literaturverzeichnis
%%%%%%%%%%%%%%%%%%%
%\usepackage[style=numeric-comp, sorting = none, backend = bibtex]{biblatex}   % Literaturverzeichnis


%%%%%%%%%%%%%%%%%%%
%Paketvorschläge Tabellen
%%%%%%%%%%%%%%%%%%%
%\usepackage{array}     % Basispaket für Tabellenkonfiguration, wird von den folgenden automatisch geladen
\usepackage{tabularx}   % Tabellen, die sich automatisch der Breite anpassen
%\usepackage{hhline}
%\usepackage{longtable} % Mehrseitige Tabellen
%\usepackage{xltabular} % Mehrseitige Tabellen mit anpassarer Breite
\usepackage{booktabs}   % Verbesserte Möglichkeiten für Tabellenlayout über horizontale Linien

%%%%%%%%%%%%%%%%%%%
%Paketvorschläge Mathematik
%%%%%%%%%%%%%%%%%%%
\usepackage{mathtools} % erweiterte Fassung von amsmath
\usepackage{amssymb}   % erweiterter Zeichensatz
%\usepackage{siunitx}   % Einheiten

%Formatierungen für Beispiele in diesem Dokument. Im Allgemeinen nicht notwendig!
\let\file\texttt
\let\code\texttt
\let\tbs\textbackslash


\usepackage{pifont}% Zapf-Dingbats Symbole
\newcommand*{\FeatureTrue}{\ding{52}}
\newcommand*{\FeatureFalse}{\ding{56}}

\usepackage{afterpage}

\newcommand\blankpage{
	\null
	\thispagestyle{empty}
	%\addtocounter{page}{-1}
	\newpage
}
\usepackage[numbers,sort&compress,square]{natbib}

\usepackage{float}
\usepackage{placeins}
\usepackage[singlelinecheck=false]{caption}
\usepackage[toc,page]{appendix}
\usepackage{cancel}
\usepackage{hyperref}
\def\UrlBreaks{\do\/\do-}
\usepackage{breakurl}

\newcommand\numberthis{\addtocounter{equation}{1}\tag{\theequation}}
\newcommand{\RN}[1]{%
	\textup{\uppercase\expandafter{\romannumeral#1}}%
}
\newcommand{\subtag}[1]{\tag{\theparentequation#1}}


\begin{document}

\Metadata{
	title=Optimizing X-Ray Production with Nanosecond Laser Pulses by Two-Plasmon Decay
}

\title{Optimizing X-Ray Production with Nanosecond Laser Pulses by Two-Plasmon Decay}
\subtitle{Optimierung der Röntgenerzeugung mit Nanosekunden-Laserpulsen durch Two-Plasmon Decay}
\author[P. Hesselbach]{Philipp Mathias Hesselbach}%optionales Argument ist die Signatur,
\birthplace{Aschaffenburg}%Geburtsort, bei Dissertationen zwingend notwendig
\reviewer{Priv. Doz. Dr. Vincent Bagnoud \and Priv. Doz. Dr. Paul Neumayer}%Gutachter

%Diese Felder werden untereinander auf der Titelseite platziert.
%\department ist eine notwendige Angabe, siehe auch dem Abschnitt `Abweichung von den Vorgaben für die Titelseite'
\department{phys} % Das Kürzel wird automatisch ersetzt und als Studienfach gewählt, siehe Liste der Kürzel im Dokument.
\institute{Institute of Nuclear Physics}
\group{Working Group Bagnoud}

\submissiondate{\today}
\examdate{\today}

%	\tuprints{urn=1234,printid=12345,doi=10.25534/tuprints-1234}
%	\dedication{Für alle, die \TeX{} nutzen.}

\maketitle

\affidavit% oder \affidavit[digital] falls eine rein digitale Abgabe vorgesehen ist.
\mbox{} \thispagestyle{empty}
\tableofcontents


\include{ChapIntroduction}
\include{ChapXrayDiffraction}
\include{ChapIncreasingNumbersOfPhotons}
\include{ChapTPD2}
\include{ChapExperimentalSetup}
\include{ChapEvaluation2}
\include{ChapResults2}
\include{ChapConclusion}

\newpage
\appendix
\addappheadtotoc

\input{ChapIntensityCalc}
\input{ChapLinesMo}

%\printbibliography
%\newpage
%\mbox{} \thispagestyle{empty}

\bibliographystyle{custom7}
\urlstyle{rm}
\bibliography{DEMO-TUDaBibliography}
\end{document}
